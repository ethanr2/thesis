\documentclass[a4paper,man,floatsintext,natbib]{apa6}
\usepackage[english]{babel}
\usepackage[utf8]{inputenc}
\usepackage[T1]{fontenc}
\usepackage{amsmath}
\usepackage{amsfonts}
\usepackage{amssymb}
\usepackage{graphicx}
\usepackage{changepage} 
\usepackage{systeme}
\usepackage{mathtools}
\usepackage{mathrsfs}
\usepackage{setspace}

\newcommand\setItemnumber[1]{\setcounter{enumi}{\numexpr#1-1\relax}}

\newcommand{\R}{\mathbb{R}}
\newcommand{\N}{\mathbb{N}}
\newcommand{\C}{\mathbb{C}}
\newcommand{\Z}{\mathbb{Z}}


%\usepackage[maxcitenames=3,style=apa]{biblatex}
%\addbibresource{refs.bib}

\doublespacing

\title{Measuring the Fed Information Effect}
\author{Ethan Rahman}
\affiliation{Northern Illinois University - ECON 592}
\shorttitle{Research Paper}


\begin{document}
	\maketitle
	\section{Introduction}
	Monetary policy is one of the most powerful options in the economic stabilization policy toolkit. Monetary policy affects just about every single part of the economy in some way. Central bankers are almost always faster than fiscal policy makers when it comes to responding to changes in macroeconomic conditions. Interest rates will be cut before congress even starts to talk about fiscal stimulus. It is this interaction between central bankers and the economy that poses a problem for those that study the effects of monetary policy. Interest rate cuts can stimulate real output, but central bankers will only want lower interest rates when real output is low. The result is simultaneity bias. This makes it very difficult to answer questions like "How much will unemployment change if the Federal Reserve hikes interest rates?" or "does monetary policy increase wealth inequality?" Any attempt to estimate these effects directly with simple regressions will always produce biased estimates because interest rates are fundamentally endogenous.\\
	
	In order to analyze the effect of monetary policy on the economy, it is crucially important to identify exogenous variation in interest rates. There have been numerous attempts to identify exogenous variation in interest rates. Approaches based on financial asset prices and high frequency identification lie at the cutting edge of the research frontier. For instance, \citeauthor{Gurkaynak2011} (2011, henceforth GSS) uses the price of federal funds rate (FFR) futures contracts around a 30 minute window of each Federal Open Market Committee (FOMC) press release announcing monetary policy decisions. If the FOMC announces an unexpected policy change, the price of the FFR futures should change accordingly. Standard macroeconomic theory suggests that market behavior should only change if policy decisions are unexpected, thus we have a source of exogenous variation. \\ 
	
	More recent research has cast doubt on these "market-surprise" type indicators. \citeauthor{Nakamura2018} (\citeyear{Nakamura2018}) find evidence for the "Fed-information effect" - the Fed's policy announcements could reveal information about the state of the economy not currently available to market participants. This raises concerns about endogeneity because a portion of the price change could simply reflect income effects rather than pure monetary policy shocks. I would like to explore the Fed-information effect by investigating what kinds of monetary policy indicators are susceptible to it.
	\section{Literature Review}
	The bulk of my research topic focuses on comparing the monetary shock indicators derived by \cite{Romer2004} with that of \cite{Gertler2015}. The Romer and Romer (henceforth RR04) indicator is derived using the Federal Reserves internal forecast of various economic state variables while the Gertler and Karadi (henceforth GK15) indicator is derived from implicit market forecasts of interest rates. Occasionally, these two indicators move in opposite directions, and my hypothesis is that these divergences could be explained in part by the "Fed-information effect" as described by \cite{Bauer2020}.
	
	RR04 exploits the Federal Reserve's Greenbook forecasts. These forecasts are made before each Federal Open Market Committee (FOMC) meeting and they serve an important role in policy decisions. The indicator is derived by running the following regression: 
	\begin{align*}
		\Delta ff_m = &\alpha + \beta ffb_m + \sum_{i=-1}^{2} \gamma_i \widetilde{\Delta y}_{mi} + \sum^2_{i=-1} \lambda_i \left(\widetilde{\Delta y}_{mi}-\widetilde{\Delta y}_{m-1,i}\right) \tag{1} \label{eq1}\\
		&+\sum^{2}_{i=-1} \phi_i \tilde{\pi}_{mi} + \sum^2_{i=-1} \theta_i \left(\tilde{ \pi}_{mi}-\tilde{ \pi}_{m-1,i}\right) + \rho \tilde{u}_{m0} + \epsilon_m
	\end{align*}
	Where \(\Delta ff_m\) is the change in interest rates decided at FOMC meeting \(m\), \(ffb_m\) is the level of interest rates before meeting \(m\) takes place, \(\widetilde{\Delta y}\) is real GDP growth, \(\tilde{\pi}\) is inflation, and \(\tilde{u}\) is unemployment. For each forecast variable \(\tilde{x}_{mi}, i\) indicates the quarter the variable is forecasted over relative to the quarter contemporaneous to meeting \(m\). For example, \(\tilde{\pi}_{m2}\) is the forecast for inflation two quarters ahead of the current quarter. \(\tilde{\pi}_{m0}\) represents a nowcast of inflation for the current quarter because we cannot observe quarterly data in real time. \(\tilde{\pi}_{m,-1}\) is observed inflation from the previous quarter, and this variable is never an actual forecast because the true observation will be available during meeting \(m\). But the most important variable in equation \ref{eq1} is \(\epsilon_m\):  the error term. It represents changes in interest rates not explained by the Fed's internal forecasts of the economy. \(\epsilon_m\) is the RR04 indicator. 
	\begin{figure}
		\centering
		\includegraphics[width=\textwidth]{charts/rr04.png}
		\caption{\label{rr04} The \cite{Romer2004} monetary policy shock time series.}
	\end{figure}
	Note that this model is not intended to be structural and we should not be too interested in the parameter estimates. We are only using the regression to purge endogenous variation in interest rates. RR04 leverage their indicator to estimate an impulse response function describing the effect of a 1\% interest rate hike on a measure of real output and the price level. Essentially, this is an attempt to describe a dynamic income-savings curve in a standard New Keynesian model. 
	
	GK15 uses a more modern identification strategy. GK15 analyzes the change in the price of Federal Funds Rate (FFR) futures contracts over a 30 minute window around every FOMC press release. Information about monetary policy is disproportionately released to the public in these tight windows of time. 10 minutes before any FOMC decision is announced, the price of FFR futures should reflect market expectations of interest rates conditional on all the information available to market participants at that time. 20 minutes after the announcement, market participants will update their forecasts conditional on the new information contained in the announcement. Taking the difference between these two prices yields a measure of unexpected changes in interest rates. If the Fed hikes rates, firms will have no reason to change their production plans if they already planned for rate hikes. Standard macroeconomic theory suggests that market actors will only change their behavior in response to unexpected changes in interest rates. Therefore, GK15's methodology allows us to identify exogenous variation in interest rates.
	\begin{figure}
		\centering
		\includegraphics[width=\textwidth]{charts/gk15.png}
		\caption{\label{gk15} The surprise in the 3 month ahead FFR futures rate \citep{Gertler2015}.}
	\end{figure}
	
	Like RR04, GK15 estimates the dynamic IS curve to determine the effect of interest rate hikes on real output over time. RR04 uses two different approaches to do so: a time series model and a vector auto-regression (VAR). GK15 only uses a VAR. As such, the VAR estimates will provide the most consistent comparison between the two indicators. In response to a 1\% increase in interest rates, the RR04 indicator implies a peak impact of -2.9\% real output growth at 23 months while the baseline GK15 indicator implies a peak impact of -2.0\% at 15 months. Consistent with macroeconomic intuition on the long run neutrality of money, the impact of both indicators declines to zero eventually. In principle, these indicators should be measuring the same thing, so a question arises: what explains the difference?  
	
	\cite{Bauer2020} may shed some light. They analyze evidence for and against what they call the Fed-information effect. They argue that indicators based on high frequency financial market "surprises" could be contaminated by omitted variable bias if the FOMC releases information about the state of the economy with each of their announcements. They analyze the effect of the \cite{Nakamura2018} indicator (very similar to GK15) on private sector forecasts of state variables. Bauer and Swanson's position is that there is weak evidence for the Fed information effect and instead provide an alternative theory that the Fed is responding to news at the same time market participants are. Among many other tests, they regress high frequency changes in the price of S\&P500 ETFs on the Nakamura indicator and test for a statistically significant negative slope. If markets react positively to rate hikes, then that is strong evidence for the Fed information effect. With this particular test, they find no evidence for the Fed-information effect. 
	
	I believe there are serious problems with this approach. On theoretical grounds, it assumes some form of irrational expectations. Market participants have to be persistently "tricked" in order for their hypothesis to work. Moreover, I do not believe the S\&P500 test is sufficient to reject the possibility of endogeneity or even the possibility that the Fed information effect is large. Omitted variable bias will cause our parameter estimates to be inconsistent and biased, but we cannot know how large the bias will be just by running the endogenous regression. 
	
	That being said, Bauer and Swanson's approach provides a starting point. The strength of the RR04 indicator is that it purges the Fed's private information about the economy away from changes in interest rates. The Fed-information effect should not be a problem for this indicator. The strength of the GK15 indicator is that it purges private sector information about monetary policy away from changes in interest rates. The GK15 indicator has a more solid theoretical justification based on micro-foundations and rational expectations. Even if the Fed's information about the economy is better than the private sector's, we are interested in understanding the behavior of the private sector. Market actors only make decisions based on the information they have. If the GK15 indicator is contaminated by the Fed-information effect, we can potentially combine the two approaches by replacing the \(\Delta ff_m\) term in equation \ref{eq1} with the GK15 indicator. A new, purified indicator can be derived. With this new indicator, we can repeat the \cite{Bauer2020} regression for both the purified indicator and the baseline GK15 indicator. A difference of slopes test can then be used to determine if the Fed-information effect actually exists. 
	\section{Economic Theory}
	The Federal Reserve conducts monetary policy primarily through changes in certain benchmark interest rate targets such as the Federal Funds rate. However, rational agents will anticipate the Fed's behavior and make decisions based on their forecasts of the Federal Reserve's future interest rate targets. Therefore, a cut in interest rates may not have any stimulative effect on the economy at all if this rate cut was fully anticipated. The Fed's objective function is based on its forecasts of various state variables in the economy such as unemployment, real output growth, and inflation. Agents will attempt to forecast these same state variables in order to predict the Fed's interest rate target. But markets will only change their behavior they are "surprised" by the Fed's interest rate target. 
	
	As others in the literature have argued \citep{Nakamura2018}, it is very possible that the Fed has information about the economy not available to the public. This creates a potential problem. Even if the Federal Open Market Committee announces an unexpected rate cut, the private sector could be "surprised" just because the FOMC released new information about its forecasts of state variables. For instance, if the FOMC announces an unexpected rate cut but also reveals that it expects unemployment to be much lower than what the private sector expects, the unexpected rate cut may not be expansionary. The rate cut would essentially be explained by the income effect or the Fisher effect, but we want to identify interest rate changes caused by the liquidity effect only. 
	
	In more precise terms, we may model the Fed's nominal interest rate target \(i\) corresponding to an FOMC meeting \(m\) as:
	\begin{align}
		i_m = i_m^p(\text{PubInfo}_m) + X_m(\text{FedInfo}_m)^\prime \alpha + \epsilon_m \label{eq1}
	\end{align}
	Where \(i_m^p\) is the private sector's forecast of \(i_m\) given all publicly available information \(\text{PubInfo}_m\) \textit{before} the policy decision is announced, \(X_m\) is a vector of the Fed's forecast of state variables given its private information \(\text{FedInfo}_m\), and \(\epsilon_m\) is an exogenous monetary policy shock. Many in the literature have argued that we may exploit the price of Federal Funds rate futures contracts to estimate \(\epsilon_m\) \citep{Gurkaynak2011,Gertler2015, Nakamura2018}. Adjusting equation \ref{eq1}:
	\begin{align*}
		i_m - i_m^p(\text{PubInfo}_m) = X_m(\text{FedInfo}_m)^\prime \alpha + \epsilon_m \\
		mps_m = X_m(\text{FedInfo}_m)^\prime \alpha + \epsilon_m \tag{2} \label{model}
	\end{align*}
	Where \(mps_m\) is the change in the price of FFR futures contracts over a 30 minute window around the policy announcement corresponding to meeting \(m\). \(mps_m\) is used as an estimate of \(\epsilon_m\), leaving \(X_m\) out as an omitted variable. \cite{Bauer2020} investigate the claim that \(X_m\) is correlated with \(mps_m\). If true, this could imply any attempt to use \(mps_m\) to estimate the effect of monetary policy on some dependent variable of interest \(y_m\) will suffer from omitted variable bias:
	\begin{align*}
		y_m &= \beta_0 + \beta_1 \epsilon_m  + v \\
		y_m &= \beta_0 + \beta_1(mps_m - X_m(\text{FedInfo}_m)^\prime \alpha )  + v \\
		y_m &= \beta_0 + \beta_1mps_m - \beta_1 X_m(\text{FedInfo}_m)^\prime \alpha + v \\
		y_m &= \beta_0 + \beta_1mps_m + u \tag{3} \label{eq3} \\ 
		\mathrm{Cov}(mps_m, u) &\neq 0 
	\end{align*}
	This is source of endogeneity is the Fed-information effect.
	\section{Data}
	We have several sources of time-series data. To measure and test the existence of the Fed information effect, we must first derive a "purified" version of \(mps_m\). We will essentially estimate equation \ref{model} and then use the residual \(\hat{\epsilon}_m\) as our indicator of monetary policy shocks. For \(X_m\), we use data from the Federal Reserve's Greenbook forecasts. These forecasts are made publicly available with a lag of 5 years. The forecasts are made before every scheduled FOMC meeting and they inform policy decisions. The forecasts are made under the assumption of no monetary policy shocks. For \(mps_m\) itself, our sample is limited because the data is proprietary. \cite{Gertler2015} have graciously made their data publicly available. We will use the change in the forecast of interest rates implied by three month ahead FFR futures. Each observation of \(mps_m\) is based on the difference between the futures price 20 minutes after the policy decision of FOMC meeting \(m\) is announced and 10 minutes before the announcement. We use a model very similar to that of \cite{Romer2004}:
	\begin{align*}
		mps_m =  \alpha &+ \sum_{i=0}^{2} \gamma_i \widetilde{\Delta y}_{mi} + \sum^2_{i=0} \lambda_i \left(\widetilde{\Delta y}_{mi}-\widetilde{\Delta y}_{m-1,i}\right) \tag{4} \label{ep_hat_model} \\
		&+\sum^{2}_{i=0} \phi_i \tilde{\pi}_{mi} + \sum^2_{i=0} \theta_i \left(\tilde{ \pi}_{mi}-\tilde{ \pi}_{m-1,i}\right) + \rho \tilde{u}_{m0} + \epsilon_m
	\end{align*}
	For variable definitions see table \ref{gbvars}. Several adjustments are made to the Romer and Romer model. All lagged terms are excluded because this represents information already available to the public, but this information should already be purged from \(mps_m\) by assumption. Market prices reflect all information publicly available. Similarly, we exclude the \(ffrb_m\) term because this is also public information and there is little evidence of mean reversion in \(mps_m\). 
	\begin{table}[ht]
		\centering
		\begin{tabular}{p{0.30\linewidth}  p{0.6\linewidth}}
			\toprule
			Variable & Description \\ 
			\midrule  
			\(m\) & FOMC meeting corresponding to the Greenbook forecast and interest rate shock in question. \\
			\(i\) & Forecast horizon relative to meeting \(m\). Data for the current quarter cannot be observed in real time, so \(i=0\) denotes a nowcast. \\
			\(\widetilde{\Delta y}_{mi}\) &  Real GDP growth forecast.\\
			\(\tilde{\pi}_{mi}\)& Inflation forecast.\\
			\(\tilde{u}_{m0}\) & Unemployment nowcast. \\
			\bottomrule
		\end{tabular}
		\caption{Definition of all variables from the Greenbook forecasts in equation \ref{ep_hat_model}. For each difference term \(x_{mi}-x_{m-1,i}\), we are taking a forecast of \(x\) at the \(i\)th quarter ahead of FOMC meeting \(m\) and looking at how the forecast for \(x\) at quarter \(i\) changed since meeting \(m-1\). In other words, we adjust the forecast horizons for meetings \(m\) and \(m-1\) so that the forecasts refer to the same quarter.}
		\label{gbvars}
	\end{table}
	
	After obtaining our indicator, we will then need to estimate the effect of monetary policy on some dependent variable. Typically, monetary policy shock indicators are used to estimate the dynamic effects on real output or inflation. But \cite{Bauer2020} provide a compelling alternative: the change in the logarithm of S\&P500 stock market price index around each FOMC meeting. Indeed, Bauer and Swanson estimate equation \ref{eq3} directly. Unfortunately, due to data limitations we cannot use high frequency prices for this process. The best we can do is use 24 hour changes in the price of the S\&P500. This is not ideal because the change in prices could be contaminated by non-FOMC news such as a BLS job report released on the same day as an FOMC announcement. The findings in \cite{Gurkaynak2011} indicate that for changes in the price of a wide variety of assets, the difference between a 30 window and a 24 hour window is small overall. Furthermore, this would really only cause problems for our purposes if the confounders are correlated with our regressors. 24 hour data on S\&P500 prices is available from \textit{Yahoo Finance}.
	\section{Empirical Analysis}
	There are multiple tests we will run to evaluate the Fed-information effect hypothesis. As mentioned previously, \cite{Bauer2020} estimate the following equation, which will act as our restricted model:
	\[
	\Delta \log{(\text{S\&P500}_m)} = \beta^R_0 + \beta^R_1 mps_m + u \tag{5} \label{rest}
	\]
	Bauer and Swanson argue that a statistically significant slope implies that the Fed information effect does not exist, however we disagree with this interpretation. What we actually need to know is how much the slope coefficient changes when we include the potentially omitted variable: \(\hat{i}_m\) which is imputed from the regression of equation \ref{ep_hat_model}. We will conduct 3 tests based on equation \ref{rest} and also the following two equations:
	\begin{align*}
		\Delta \log{(\text{S\&P500}_m)} &= \beta_0 + \beta_1 mps_m + \beta_2 \hat{i}_m + v \tag{6} \label{unrest}\\
		\Delta \log{(\text{S\&P500}_m)} &= \delta_0 + \delta_1 \hat{\epsilon}_m + w \tag{7} \label{trumod}
	\end{align*}
	Here, equation \ref{trumod} is the most direct representation of the "true model" if the Fed information effect is statistically significant. Note that the exclusion of \(\hat{i}_m\) from equation \ref{trumod} should not make a difference because \(\mathrm{Cov}(\hat{i}_m, \hat{\epsilon}_m) = 0\) by construction.
	
	\subsubsection{Student's T test} Let \(\hat{\beta}_1^R\) denote the OLS estimate of \(\beta^R_1\) from equation \ref{rest}.
	\begin{align*}
		H_0: \beta_1 &= \hat{\beta}_1^R \\
		H_a: \beta_1 &> \hat{\beta}_1^R \\
		\text{Test statistic: } t &= \frac{\hat{\beta}_1 -\hat{\beta}_1^R}{\mathrm{se}(\beta_1)} \sim {\mathrm{T}}(n-2) 
	\end{align*}
	This test will determine whether the alternative specification changes the slope coefficient on \(mps_m\) significantly. While this test is straightforward, it does not capture any information about the coefficient for \(\hat{i}_m\).
	\subsubsection{Wald Test}
	Define the restriction matrix \(R =\begin{pmatrix}
		0 &1  &0\\
		0 &0  &1\\
	\end{pmatrix}\), our test is then:
	\begin{align*}
		H_0: R\beta &=\begin{pmatrix}
			\hat{\beta}^R_1 \\
			0 \\
		\end{pmatrix} \\
		H_a:  R\beta &\neq\begin{pmatrix}
			\hat{\beta}^R_1 \\
			0 \\
		\end{pmatrix}\\
		\text{Test statistic: } W &= \left(R\hat{\beta} - \begin{pmatrix}
			\hat{\beta}^R_1 \\
			0 \\
		\end{pmatrix}\right)^\prime\left(R\cdot \frac{\mathrm{Var}(\hat{\beta})}{n}\cdot R^\prime \right)^{-1}\left(R\hat{\beta} - \begin{pmatrix}
			\hat{\beta}^R_1 \\
			0 \\
		\end{pmatrix}\right) \sim \chi^2_2
	\end{align*}
	A significant result would provide strong evidence for the Fed-information effect. Both the Wald and Student's t tests do not account for uncertainty in the estimate of \(\beta_1^R\). This can be addressed by our final test.
	\subsubsection{Wu-Hausman Test}
	\begin{align*}
		H_0: &\delta_1 \text{ and } \beta_1^R \text{ are consistent.} \\
		H_a: &\delta_1 \text{ is consistent but } \beta_1^R \text{ is inconsistent.} \\
		\text{Test statistic: } &H = \frac{(\hat{\delta}_1-\hat{\beta}_1^R)^2}{\mathrm{Var}(\hat{\delta}_1)-\mathrm{Var}(\hat{\beta}_1^R)} \sim \chi^2_1
	\end{align*}
	If none of these tests are suitable perhaps another method could involve bootstrapping the distribution of \(\delta_1 - \beta_1^R\), allowing us to verify whether \(\delta_1 - \beta_1^R > 0\).
	\section{Hypothesized Findings}
	Depending on instructor feedback, we will choose one or more of the above tests to report on the final paper. Generally speaking, we think it is highly plausible that the Federal Reserve has better information about the economy than the general public. Furthermore it would be a remarkable coincidence if new information released did not correlate at all with unexpected interest rate changes. Finally the alternative explanation offered by \cite{Bauer2020} is inconsistent with rational expectations. As a prior, we think rational expectations is a good assumption. We suspect that all the tests will point in the same direction - rejection of the null hypotheses, suggesting that the Fed-information effect exists.  
	\bibliography{refs}
\end{document}